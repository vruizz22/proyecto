\documentclass[letterpaper]{article}
\usepackage[spanish]{babel}
\selectlanguage{spanish}
\usepackage[utf8]{inputenc}

\usepackage{lipsum}
\usepackage{amsmath,amssymb,amsfonts,amsbsy}
\usepackage{array}
\usepackage{graphicx}
\usepackage{subfigure}
\usepackage{float}
\usepackage{hyperref}
\usepackage{ragged2e}

\graphicspath{ {./figures/} }

%\usepackage[pass]{geometry}
\usepackage[left=1.25in,right=1.25in,top=1.0in,bottom=1.0in]{geometry}
\usepackage{listings}

% Custom colors
\usepackage{color}
\definecolor{deepblue}{rgb}{0,0,0.65}
\definecolor{deepred}{rgb}{0.7,0,0}
\definecolor{deepgreen}{rgb}{0,0.6,0}

\newcommand{\mytitle}{Informe}
\newcommand{\myauthor}{Grupo 24}
\newcommand{\mydate}{\today}


\makeatletter
\renewenvironment{thebibliography}[1]
     {\section{\bibname}% <-- Cambiado de \chapter* a \section*
      \@mkboth{\MakeUppercase\bibname}{\MakeUppercase\bibname}%
      \list{\@biblabel{\@arabic\c@enumiv}}%
           {\settowidth\labelwidth{\@biblabel{#1}}%
            \leftmargin\labelwidth
            \advance\leftmargin\labelsep
            \@openbib@code
            \usecounter{enumiv}%
            \let\p@enumiv\@empty
            \renewcommand\theenumiv{\@arabic\c@enumiv}}%
      \sloppy
      \clubpenalty4000
      \@clubpenalty \clubpenalty
      \widowpenalty4000%
      \sfcode`\.\@m}
     {\def\@noitemerr
       {\@latex@warning{Empty `thebibliography' environment}}%
      \endlist}
\makeatother

\begin{document}

\begin{minipage}[t]{.13\textwidth}
	\vspace{-0.25in}
	\begin{figure}[H]
		\includegraphics[width=0.90\textwidth]{LogoUC.jpg}
	\end{figure}
\end{minipage}
\hfill
\begin{minipage}[t]{.85\textwidth}
	\vspace{0pt}
	\begin{flushleft}
		\begin{tabular}{l}
			{\sc Pontificia Universidad Cat\'olica de Chile}         \\
			{\sc Escuela de Ingenier\'ia}                            \\
			{\sc Departamento de Ingenier\'ia Industrial y Sistemas} \\
			{\sc ICS1113-Optimizaci\'on}
		\end{tabular}
	\end{flushleft}
\end{minipage}
\vspace{0pt}
\hfill
\vspace*{6cm}
\begin{center}{}
	\vspace*{2mm}
	{\Huge\bf Informe 2}\\
	\vspace*{4mm}
	\hrule\vspace*{1pt}\hrule
	\vspace*{4mm}
	{\LARGE\bf Optimizar la posici\'on de electrolineras y su rentabilidad para Copec}\\
	\vspace*{4mm}
	{\huge\bf Grupo 24 }\\
	\vspace*{1mm}
\end{center}

\vspace*{30mm}
\flushright

Gabriel Cornejo 23647086 Sección 1\\
Sebastián Lorca 23200316 Sección 2\\
Pablo Rojas 23645016 Sección 1\\
Benjamín Sánchez  23205873 Sección 1\\
Víctor Ruiz 2320012J Sección 1\\


\vspace*{5mm}
{\large Fecha entrega: 07 de Mayo de 2024\\}

\newpage
\begin{flushleft}
	\tableofcontents
\end{flushleft}

\newpage
\begin{flushleft}

	\section{Descripción del Problema}
	\subsection{Contexto y beneficios de resolver el problema}
	\justifying
		% Contexto -> Como Copec se enfrenta al desafío de optimizar el posicionamiento de sus centros de carga para vehículos eléctricos (CVE).
		% Mejorar la rentabilidad de estaciones de carga eléctrica para beneficio de la empresa y para masificar la adopción de vehículos eléctricos.
		La transición energética es tema mundial por la importancia de generar un cambio a corto plazo en materia ambiental. En esta línea, Chile tiene metas propuestas para reducir la huella de carbono y para ello, uno de los principales desafíos es en materia automotriz, donde los automóviles eléctricos están cada vez más presentes y se proyecta que para el año 2050 el 40 \% de los vehículos de uso particular sean eléctricos. \cite{Gobierno 1} \\
		Esta proyección se está cumpliendo, ya que se ha visto un crecimiento acorde a lo esperado. Por ejemplo, según datos del sitio Statista en Chile las ventas de autos eléctricos han aumentado considerablemente, teniendo el año 2022, 1295 ventas, que representa más del 200 \% respecto al año 2021 y más del 500\% respecto al 2019. Statista. (2023, 15 octubre). \cite{Gobierno 2} \\


		Entonces, en este contexto de transición hacia una movilidad más sostenible, Copec, una empresa chilena líder en la distribución de combustibles en América Latina, ha decidido incursionar en el mercado de vehículos eléctricos con su plan de movilidad sustentable (Copec Voltex, s.f.) \cite{copec}. Este plan consiste en la implementación de electrolineras, que son puntos de carga públicos para BEV1 a lo largo de todo Chile, para hacer posible una red conectada, donde usuarios puedan desplazarse sin depender de la autonomía de su EV2. 
		Uno de los mayores desafíos presentes en esta iniciativa es lograr optimizar el posicionamiento de sus centros de carga para vehículos eléctricos (CVE). Este proceso implica identificar las ubicaciones óptimas para instalar estos centros de carga, considerando diversos factores como la demanda potencial, la infraestructura eléctrica disponible, la accesibilidad y la rentabilidad económica. \\

		El tomador de decisiones en este caso es el equipo de planificación estratégica de Copec Voltex, que busca maximizar la rentabilidad de los centros de carga eléctricos en Chile, satisfaciendo la creciente demanda de vehículos eléctricos. El horizonte de planificación adecuado abarca al menos un período de 5 años, ya que se espera que la adopción de vehículos eléctricos continúe en aumento durante este tiempo. 

		Resolver este desafío le entregará a Copec Voltex la iniciativa de aumentar la cantidad de electrolineras, lo que, a su vez, no solo facilitará el acceso a esta tecnología emergente, sino que puede impulsar la adopción de esta tecnología al reducir las barreras de acceso para los conductores. De esta manera, lo que contribuirá significativamente a la reducción de emisiones contaminantes y al combate del cambio climático que es justamente el compromiso de Copec con su comunidad. \\

		Actualmente, Copec Voltex cuenta con una red de carga de 68 electrolineras en la Región Metropolitana, y 128 puntos a lo largo de todo el país. Eso significa una conexión de 1400 kilómetros, según indican en su sitio (Copec Voltex, s.f.). Sin embargo, tras un análisis de la autonomía de los EVs en promedio, y las distancias entre electrolineras en Chile, algunos EVs económicos como el Mazda MX-30 EV no logran cruzar las distancias entre electrolineras (Scheer, 2022) \cite{scheer}. Por esto, un modelo que sea capaz de encontrar una solución de red para todo vehículo se hace necesaria. \\

		Continuando en este eje, según una encuesta realizada por el diario la Tercera el 15\% de los encuestados cree que una de las barreras para adquirir un vehiculo electrico es la que la red de carga no avanza del todo de acuerdo a lo esperado \cite{Tercera}. Respecto a este desafío ya conocido Diego Pardow, el ministro de energía habló sobre la Hoja de Ruta para el Avance de la Electromovilidad en Chile, donde uno de los ejes estratégicos para incentivar el uso de vehículos eléctricos es precisamente mejorar la red de carga.
		Es en esta línea que se considera de gran valor el problema a optimizar, ya que, según Diego Mendoza, secretario general de ANAC (Asociación Nacional Automotriz de Chile) el parque electrificado a la fecha de septiembre del 2023 es de 4610 vehículos, y se espera que para el año 2025 el 5\% del mercado Automotriz sea de vehículos eléctricos. Tomando datos anteriores, en el año 2023 se comercializaron 313.865 vehículos \cite{chileautos} mientras que en el año 2022 la cifra fue de 426.772 \cite{Tercera-1}. Suponiendo que en el año 2025 en el mercado habran 350.000 vehículos en el mercado, si se cumple la proyección esperada se estarían comercializando 17.500 vehículos eléctricos aproximadamente. A su vez, asumiendo que habrá una especie de crecimiento lineal de comercialización de vehículos eléctricos y considerando los que ya hay en circulación, se puede estimar que para el año 2025 habrian alrededor de 30.000 vehículos eléctricos en circulación lo cual representa el 650\% de la cantidad actual. Siguiendo en esta línea considerando que Copec tiene la mayor implicancia en el mercado de puntos de carga para EVs para el año 2025 se habría ayudado a más de 10.000 vehículos eléctricos generando una red de puntos de carga óptima acorde al crecimiento esperado y que se está desarrollando en el país.
	\subsection{Objetivo que persigue el tomador de decisiones}
		El objetivo principal del equipo de planificación estratégica de Copec Voltex es maximizar la rentabilidad de las electrolineras, identificando las ubicaciones óptimas para instalar centros de carga para vehículos eléctricos. Esto implica tomar decisiones sobre la cantidad de centros de carga a instalar, sus ubicaciones específicas y la capacidad de carga de cada uno. Las restricciones involucradas en este proceso de decisión incluyen limitaciones presupuestarias, restricciones regulatorias y consideraciones logísticas relacionadas con la infraestructura eléctrica disponible y sus ubicaciones.
	
	\section{Modelación del problema}
	\subsection*{Supuestos}
	\begin{itemize}
		\item La cantidad de estaciones existente no supera la capacidad máxima de la infraestructura eléctrica.
		\item Los \textit{EV}s van a cargar en promedio $\delta$ KW en un mes.
		\item La demanda total se considerar\'a como la cantidad de veh\'iculos disponibles en la regi\'on.
	\end{itemize}
	\subsection{Conjuntos}
	\begin{itemize}
		\item $t \in \{1, \ldots, 60\}$, el mes desde la implementación del modelo.
		\item $i \in I$, donde $I$ es el conjunto de ubicaciones de los posibles centros de carga.
		\item $m \in M$, donde $M$ es el conjunto de tipos de cargadores.
	\end{itemize}

	\subsection{Parámetros}
	\begin{itemize}
		\item $D_{mit}$, demanda total de cargadores tipo $m$ en la estación $i$ para el periodo $t$.
		\item $CI_{t}$, el costo de instalar un centro de carga en el periodo $t$.
		\item $CP_{mt}$, el costo de comprar un cargador tipo $m$ en el periodo $t$.
		\item $CC_{mit}$, el costo de instalar un cargador tipo $m$ en la estación $i$ en el periodo $t$.
		\item $CKW_{mit}$, el costo de energía eléctrica por kilowatt-hora para un cargador tipo $m$ en la estación $i$ en el periodo $t$.
		\item $CM_{mit}$, el costo de mantención de un cargador tipo $m$ en la estación $i$ en el periodo $t$.
		\item $\alpha$, coeficiente de ganancia espera por el precio seleccionado, por KW de electricidad vendido.
		\item $\delta$, cantidad de KW que se espera que cargue un vehículo eléctrico en un mes.
		\item $\phi_m$, capacidad de carga por mes de un cargador tipo $m$ en KW.
		\item $K$, la capacidad eléctrica máxima que permite la infraestructura eléctrica.
		\item $EI_{i}$, si ya existe la infraestructura eléctrica en la estación $i$.
		\item $EC_{mi}$, la cantidad de estaciones de carga de tipo $m$ que ya existen en la estación $i$ en el mes $t$
		\item $CS_{mt}$, el costo de almacenar un cargador tipo $m$ en el periodo $t$.
	\end{itemize}
	\subsection{Variables de decisión}
	\begin{itemize}
		\item $x_{mit}$ cantidad de cargadores tipo $m$ en la estación $i$ para el periodo $t$.
		\item \[
			      y_{it} =
			      \begin{cases}
				      1 & \quad\text{si se instala la infraestructura eléctrica para }i\text{ en }t \\
				      0 & \quad\text{en cualquier otro caso.}
			      \end{cases}
		      \]
		\item \[
			      z_{it} =
			      \begin{cases}
				      1 & \quad\text{si existe la infraestructura eléctrica para }i\text{ en }t \\
				      0 & \quad\text{en cualquier otro caso.}
			      \end{cases}
		      \]
		\item $a_{mt}$, cantidad de cargadores tipo $m$ que se compran en el periodo $t$.
		\item $b_{mit}$, cantidad de cargadores tipo $m$ que se instalan en la estación $i$ en el periodo $t$.
		\item $d_{mit}$, (Cant de vehiculos)demanda que se va a satisfacer para cargadores tipo $m$ en la estación $i$ en el periodo $t$.
	\end{itemize}
	\subsection{Función Objetivo}
	% COLOCAR COSTOS DE INSTALACION carga y insgractuructura
	% Costo de mantener inventario
	\begin{center}
		$\max \; \sum_{m \in M}\sum_{i \in I} \sum_{t=1}^{60} (d_{mit} \cdot CKW_{mit} \cdot (\alpha - 1) - x_{mit} \cdot CM_{mit}) - \sum_{i \in I} \sum_{t=1}^{60} a_{mt} \cdot CP_{mt} \; $ 
	\end{center}

	\subsection{Restricciones}

	% COSTO POR INVENTARIO
	\begin{enumerate}
		\item Restricción de inventario, incluyendo la condici\'on inicial (\textit{storage}).
		      \begin{align*}
			       & S_{m(t-1)} + a_{mt} = S_{mt} + \sum_{i \in I} b_{mit} &  & \forall \; m \in M, t \in \{2, \ldots, 60\} \\
			       & a_{m1} = S_{m1} + \sum_{i \in I} b_{mi1}              &  & \forall \; m \in M
		      \end{align*}
		\item Restricción de cantidad de cargadores instalados ($x$) que solo puede ser mayor a $0$ cuando se instala la infraestructura eléctrica ($y$).
		      \begin{align*}
			       & N \cdot \sum_{t'=1}^{t} y_{it'} \geq x_{mit} &  & \forall \; m \in M, \; i \in I,\; t \in \{1, \ldots, 60\}
		      \end{align*}
		\item S\'olo se puede instalar la infraestructura el\'ectrica una vez si no se ha instalado antes ($EI$).
		      \begin{align*}
			       & \sum_{t \in T} y_{it} \leq 1 - EI_i &  & \forall \; i \in I
		      \end{align*}
		\item La capacidad en KW de los cargadores instalados ($x$) no puede superar la capacidad m\'axima de KW de la infraestructura el\'ectrica ($K$).
		      \begin{align*}
			       & K \geq \sum_{m \in M} \sum_{i \in I} x_{mit} \cdot \phi_m &  & \forall \; i \in I, \; t \in \{1, \ldots, 60\}
		      \end{align*}
		\item Solo puede haber infraestructura el\'ectrica en una ubicaci\'on ($z$) si se ha instalado anteriormente ($y$)
		      \begin{align*}
			       & z_{it} \geq y_{it} + z_{i(t-1)} &  & \forall \; i \in I, \;t \in \{2, \ldots, 60\} \\
			       & z_{i1} \geq y_{i1} + EI_i       &  & \forall \; i \in I
		      \end{align*}
		\item Solo puede haber un centro de carga en una ubicaci\'on $j$ existe al menos una estaci\'on cuya distancia es menor a la distancia m\'axima permitida ($AM$).
		      \begin{align*}
			       & \sum_{i \in I: i \neq j, d_{ij}\leq AM} z_{it} \geq 1 &  & \forall \; j \in I, \; t \in \{1, \ldots, 60\}
		      \end{align*}
		\item La cantidad de cargadores en una estaci\'on ($x$) debe ser igual a la cantidad instalada en el periodo más la existente en el periodo anterior, considerando la condici\'on inicial.
		      \begin{align*}
			       & x_{mit} = b_{mit} + x_{mi(t-1)} &  & \forall \; m \in M, \; i \in I, \; t \in \{2, \ldots, 60\} \\
			       & x_{mi1} = b_{mi1} + EC_{mi}     &  & \forall \; m \in M, \; i \in I
		      \end{align*}
		\item La demanda a satisfacer, entendida como cantidad de vehículos, no puede superar la demanda total de cargadores en una estación.
		      \begin{align*}
			       & d_{mit} \leq D_{mit} &  & \forall \; m \in M, \; i \in I, \; t \in \{1, \ldots, 60\}
		      \end{align*}
		\item La cantidad de KW que se van a proveer no puede superar la capacidad de carga de los cargadores instalados.
		      \begin{align*}
			       & \delta \cdot d_{mit} \leq x_{mit} \cdot \phi_m &  & \forall \; m \in M, \; i \in I, \; t \in \{1, \ldots, 60\}
		      \end{align*}
		\item Naturaleza de las variables.
		      \begin{align*}
			       & y_{it}, z_{it} \in \{0, 1\}                             &  & \forall \; i \in I, \; t \in \{1, \ldots, 60\} \\
			       & x_{mit}, a_{mit}, b_{mit}, d_{mit} \in \mathbb{Z}^{+}_0 &  & \forall \; m\in M, i\in I, t\in T
		      \end{align*}
	\end{enumerate}

	\subsection{Naturaleza de las variables}
	\begin{itemize}
		\item $x_{mit}$: Cantidad de cargadores tipo $m$ en la estación $i$ para el periodo $t$. Es una variable entera no negativa.
		\item $a_{mit}$: Cantidad de cargadores tipo $m$ que se compran en el periodo $t$. Es una variable entera no negativa.
		\item $b_{mit}$: Cantidad de cargadores tipo $m$ que se instalan en la estación $i$ en el periodo $t$. Es una variable entera no negativa.
		\item $d_{mit}$: Demanda que se va a satisfacer para cargadores tipo $m$ en la estación $i$ en el periodo $t$. Es una variable entera no negativa.
		\item $y_{it}$: Variable binaria que indica si se instala la infraestructura eléctrica para la ubicación $i$ en el periodo $t$. Toma el valor 1 si se instala la infraestructura y 0 en cualquier otro caso.
		\item $z_{it}$: Variable binaria que indica si existe la infraestructura eléctrica para la ubicación $i$ en el periodo $t$. Toma el valor 1 si la infraestructura existe y 0 en cualquier otro caso.
	\end{itemize}
	\subsection{Definición de datos}
	\begin{itemize}
		\item $B$ es el presupuesto total disponible.
		\item $D_{\text{max}}$ es la distancia máxima permitida para viajar desde cualquier ubicación hasta el centro de carga más cercano.
	\end{itemize}

	\newpage
	
	\begin{thebibliography}{15}
	\bibitem{statista}
	Statista. (2023, 15 octubre). Chile: volumen de ventas de vehículos eléctricos 2013-2022. Recuperado de \url{https://es.statista.com/estadisticas/1179792/volumen-ventas-vehiculos-electricos-chile/}

	\bibitem{copec}
	Nosotros. (s. f.). Tienda Copec Voltex. Recuperado de \url{https://copecvoltex.cl/pages/nosotros}
	\bibitem{scheer}
	Scheer, H. (2022). Mazda MX-30 EV no logra cruzar las distancias entre electrolineras en Chile. Recuperado de \url{https://www.scheer.cl/mazda-mx-30-ev-no-logra-cruzar-las-distancias-entre-electrolineras-en-chile/}

	\bibitem{Gobierno 1}
	Plataforma de Electromovilidad - políticas, estrategias de electromovilidad en Chile. (s. f.). Recuperado de\url{https://energia.gob.cl/electromovilidad/orientaciones-de-politicas-publicas#:~:text=Este%20documento%20fue%20el%20primero,sean%20el%2040%25%20del%20parque}
	\bibitem{Gobierno 2}
	ShieldSquare Block. (s. f.). Recuperado de \url{https://www.energia.gob.cl/sites/default/files/energia_2050_-_politica_energetica_de_chile.pdf}

	\bibitem{Enel}
	Autos Eléctricos: los números de la electromovilidad en Chile. (s. f.). Enel X. \url{https://www.enelx.com/cl/es/historias/autos-electricos-el-futuro-de-la-electromovilidad-en-chile}

	\bibitem{UCH}
	Luis, V. D., De Souza Antonio, Z., Rodrigo, M. V., Sebastián, M. A., \& Manuel, P. D. V. (2022). Metodologías de proyección de demanda y evaluación del impacto de vehículos eléctricos en redes de distribución. \url{https://repositorio.uchile.cl/handle/2250/187735}

	\bibitem{paredes}
	Paredes, G. (2022).

	\bibitem{Tercera}
	Gil, L. G. (2023, 8 septiembre). ¿Cuántos hay? ¿Cuánto valen? ¿Es factible? Radiografía al mercado de los autos eléctricos e híbridos en Chile. La Tercera. Recuperado de \url{https://www.latercera.com/mtonline/noticia/cuantos-hay-cuanto-valen-es-factible-radiografia-al-mercado-de-los-autos-electricos-e-hibridos-en-chile/KXXHOYPUSNHE3HOQ6J4VKOEYOM/#}
	
	\bibitem{chileautos}
	Loyola, A. (2024, 9 enero). Venta de autos nuevos registra caída de un 26,5\% en el 2023. Recuperado de \url{https://www.chileautos.cl/noticias/detalle/venta-de-autos-nuevos-registra-caida-de-un-265-en-el-2023--29449/#:~:text=Informe%20de%20ANAC%20se%C3%B1al%C3%B3%20que,5%25%20respecto%20del%20a%C3%B1o%20anterior}

	\bibitem{Tercera-1}
	Gil, L. G. (2023a, enero 5). Chile logra récord histórico en venta de autos nuevos y nuevamente es el segundo mayor mercado de Sudamérica. La Tercera. Recupera de \url{https://www.google.com/amp/s/www.latercera.com/mtonline/noticia/venta-de-autos-supera-las-420-mil-unidades-y-chile-nuevamente-es-el-segundo-mayor-mercado-de-sudamerica/GHIJNCAWTBF5XFJSCYCGKGMNKA/%3foutputType=amp}
	\end{thebibliography}

	\newpage
	\section{Anexos}
	\begin{figure}[htbp]
			\centering
			\includegraphics[scale=0.4]{grafica-1}
			\caption{Número de vehículos eléctricos vendidos en Chile entre 2013 y 2022. \cite{statista}}
			\label{fig:grafica1}
		\end{figure}
	
	\begin{figure}[htbp]
			\centering
			\includegraphics[scale=0.5]{grafica-2}
			\caption{Número de autos privados \cite{paredes}}
			\label{fig:grafica2}
		\end{figure}

	\end{flushleft}
\end{document}
