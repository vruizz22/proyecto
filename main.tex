\documentclass[letterpaper]{article}
\usepackage[spanish]{babel}
\selectlanguage{spanish}
\usepackage[utf8]{inputenc}

\usepackage{lipsum}
\usepackage{amsmath,amssymb,amsfonts,amsbsy}
\usepackage{array}
\usepackage{graphicx}
\usepackage{subfigure}
\usepackage{float}
\usepackage{hyperref}
\usepackage{ragged2e}

\graphicspath{ {./figures/} }

%\usepackage[pass]{geometry}
\usepackage[left=1.25in,right=1.25in,top=1.0in,bottom=1.0in]{geometry}
\usepackage{listings}

% Custom colors
\usepackage{color}
\definecolor{deepblue}{rgb}{0,0,0.65}
\definecolor{deepred}{rgb}{0.7,0,0}
\definecolor{deepgreen}{rgb}{0,0.6,0}

\newcommand{\mytitle}{Tarea 1}
\newcommand{\myauthor}{Mariana Ortega}
\newcommand{\mydate}{\today}

\begin{document}

\begin{minipage}[t]{.13\textwidth}
	\vspace{-0.25in}
	\begin{figure}[H]
	\includegraphics[width=0.90\textwidth]{LogoUC.jpg}
	\end{figure}
\end{minipage}
\hfill
\begin{minipage}[t]{.85\textwidth}
    \vspace{0pt}
    \begin{flushleft}
      \begin{tabular}{l}
	{\sc Pontificia Universidad Cat\'olica de Chile}\\
  	{\sc Escuela de Ingenier\'ia}\\
  	{\sc Departamento de Ingenier\'ia Industrial y Sistemas}\\
 	 {\sc ICS1113-Optimizaci\'on}
 \end{tabular}
	\end{flushleft}
\end{minipage}
\vspace{0pt}
\hfill
\vspace*{6cm}
\begin{center}{}
\vspace*{2mm}
{\Huge\bf Informe 1}\\
\vspace*{4mm}
\hrule\vspace*{1pt}\hrule
\vspace*{4mm}
{\LARGE\bf Optimizar la posición de estaciones de carga eléctrica para vehículos}\\
\vspace*{4mm}
{\huge\bf Grupo 24 }\\
\vspace*{1mm}
\end{center}

\vspace*{30mm}
\flushright 
	
Nombre integrante 1	número de alumno 1 sección alumno 1 \\
Nombre integrante 2	número de alumno 2 sección alumno 2\\
Nombre integrante 3	número de alumno 3 sección alumno 3\\
Nombre integrante 4	número de alumno 4 sección alumno 4\\
Victor Ruiz	2320012J sección 5\\
Nombre integrante 6 número de alumno 6 sección alumno 6\\

 
\vspace*{5mm}
{\large Fecha entrega: XX de XX de 202X\\}

\newpage
\begin{flushleft}
\tableofcontents
\end{flushleft}

\newpage
\begin{flushleft}
\section{Descripción del Problema 3}
\subsection{Contexto y beneficios de resolver el problema}
\begin{quote}
    En el contexto actual de transición hacia una movilidad más sostenible, Shell, una empresa líder en la distribución de combustibles en América Latina, ha decidido incursionar en el mercado de vehículos eléctricos. Como parte de esta iniciativa, se enfrenta al desafío de optimizar el posicionamiento de sus centros de carga para vehículos eléctricos (CVE). Este proceso implica identificar las ubicaciones óptimas para instalar estos centros de carga, considerando diversos factores como la demanda potencial, la infraestructura eléctrica disponible, la accesibilidad y la rentabilidad económica. 

    El tomador de decisiones en este caso es el equipo de planificación estratégica de Shell, que busca maximizar la eficiencia de su red de centros de carga para satisfacer la creciente demanda de vehículos eléctricos. El horizonte de planificación adecuado abarca al menos un período de 5 años, ya que se espera que la adopción de vehículos eléctricos continúe en aumento durante este tiempo. 

    Resolver esta problemática es de suma importancia para Shell por varias razones. En primer lugar, la correcta ubicación de los centros de carga para vehículos eléctricos puede impulsar la adopción de esta tecnología al reducir las barreras de acceso para los conductores, lo que contribuirá significativamente a la reducción de emisiones contaminantes y al combate del cambio climático. 

    Además, optimizar el posicionamiento de estos centros puede tener un impacto económico significativo para Shell. Al maximizar la eficiencia de la red de carga, la empresa puede aumentar sus ingresos al atraer a más clientes y al mismo tiempo reducir los costos operativos asociados con la gestión de la infraestructura de carga. 

\end{quote}

\subsection{Objetivo que persigue el tomador de decisiones}
\begin{quote}
    El objetivo principal del equipo de planificación estratégica de Shell es identificar las ubicaciones óptimas para instalar centros de carga para vehículos eléctricos, de modo que se maximice la cobertura de la red, se minimicen los tiempos de espera y se optimice la rentabilidad económica. Esto implica tomar decisiones sobre la cantidad de centros de carga a instalar, sus ubicaciones específicas y la capacidad de carga de cada uno. Las restricciones involucradas en este proceso de decisión incluyen limitaciones presupuestarias, restricciones regulatorias y consideraciones logísticas relacionadas con la infraestructura eléctrica disponible. 
\end{quote}

\section{Modelación del problema 4}
\subsection{Conjuntos}
\begin{itemize}
    \item $t \in \{1, \ldots, 5\}$, el año desde la implementación del proyecto.
\end{itemize}
\subsection{Parámetros}
\begin{itemize}
    \item $i \in \{1, \ldots, N\}$, la ubicación de un centro de carga.
    \item $j \in \{1, \ldots, N\}$, la ubicación de otro centro de carga.
    \item $N$, el número de posibles ubicaciones para los centros de carga.
\end{itemize}
\subsection{Variables de decisión} 
\begin{itemize}
    \item $x_i$, una variable binaria que indica si se instala un centro de carga en la ubicación $i$ ($x_i = 1$) o no ($x_i = 0$).
    \item $C_i$, una variable continua que da la cantidad de cargadores por estación.
    \item $D_{it}$, una variable que representa la demanda en la estación $i$ para el periodo $t$.
\end{itemize}
\subsection{Función Objetivo} 
\subsection{Restricciones} 
\subsection{Naturaleza de las variables} 

\section{Definición de datos} 
\section{Resolución del problema} 
\section{Validación del resultado} 
\section{Análisis de sensibilidad} 
\section{Conclusión} \dots 14
\end{flushleft}

\subsection{Contexto y beneficios de resolver el problema} 





\end{document}